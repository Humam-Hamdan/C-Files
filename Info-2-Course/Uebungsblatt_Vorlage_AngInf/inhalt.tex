
%%%%%%%%%%%%%%%%%%%%%%%%%%%%%%%%%%%%%%%%%%%%%%%%%%%%%%
%% Insert your solutions here %%%%%%%%%%%%%%%%%%%%%%%%
%%%%%%%%%%%%%%%%%%%%%%%%%%%%%%%%%%%%%%%%%%%%%%%%%%%%%%

\Aufgabe{1}
Hilfreiche Links:
\begin{itemize}
    \item Latex-Tutorial der Uni Graz: 
    \url{https://latex.tugraz.at/latex/tutorial}
    \item LaTeX-Symbol Liste: \url{https://ctan.net/info/symbols/comprehensive/symbols-a4.pdf}
    \item LaTeX-Symbol Suche: \url{http://detexify.kirelabs.org/classify.html}
\end{itemize}

\Aufgabe{2}
\begin{enumerate}[a)] %Ersetze [a)] durch [i)] oder gar durch [test3 1 testasdf]
\item Aufzählungen und Auflistungen lassen sich mittels der enumerate bzw. itemize Umgebung realisieren. Weitere Infos \& Beispiele: 
\begin{itemize}
\item \url{https://www.latex-tutorial.com/tutorials/lists/}
\end{itemize}
\item Der Mathemodus:
\begin{itemize} \item Mathematische Symbole und Formeln lassen sich einfach mit zwei \$-Zeichen den Text integrieren: $x = -1$. Eine Formel lässt sich mittels doppelter \$-Zeichen absetzen: $$ x = -1 \Rightarrow x^2 = 1.$$
    \item Für mehrzeilige Formeln empfiehlt sich die align* Umgebung. Eine align* Umgebung gliedert sich in mehrere Spalten, welche mittels \&-Zeichen getrennt werden. Hiermit können wir zum Beispiel erzwingen, dass die Gleichheitszeichen untereinander stehen.
  \begin{align*}
      \sqrt{\frac{16}{36}} &= \frac{4}{6}  \\
     &= \frac{2}{3}
  \end{align*}
  \item 
 Für (schöne) Äquivalenzumformungen sollte zusätzlich die array Umgebung verwendet werden. Dieser Umgebung teilt man die Anzahl der Spalten sowie dessen Ausrichtung  mit:
    \begin{align*}
        \begin{array}{lrcl} % 4 Spalten da 4 Buchstaben im Argument der Array Umgebung. Die Ausrichtung der Spalten ist also (1. Spalte: links, 2. Spalte: rechts, 3. Spalte center (mittig), 4. Spalte links.
            & x+1 &=& 17 \\
            \Longleftrightarrow & x &=& 16
        \end{array}
    \end{align*}
  \item
  Im Gegensatz zur align* Umgebung, erhält in der align Umgebung jede Zeile eine Zahl:
  \begin{align}
      y &= 6  \\
      x &= 7 \label{GleichungX7}
  \end{align}
  Sofern wir der Zeile mittels des \texttt{\textbackslash{}label\{labelname\}} Befehls einen Namen gegeben haben, lässt sich diese Zeile mit \texttt{\textbackslash{}ref\{labelname\}} referenzieren: \cref{GleichungX7}.
\end{itemize}
\end{enumerate}

\Aufgabe{3}
LaTeX bietet auch Möglichkeiten für Pseudocodes. Diese Vorlage benutzt für die Umgebung \texttt{algorithm} das Package \texttt{algorithm2e}.
\begin{itemize}
\item Dokumentation: \url{http://mirrors.ctan.org/macros/latex/contrib/algorithm2e/doc/algorithm2e.pdf}
\end{itemize}

\begin{algorithm}[H]
\SetAlgoLined
\DontPrintSemicolon
\KwIn{dieser Text}
\KwOut{ein Algorithmus in \LaTeX}
\BlankLine
initialisiere \tcc*{Kommentar zur aktuellen Zeile}
\While{noch nicht am Ende des Dokuments}{
    lies aktuellen Abschnitt\;
    \eIf{verstanden}{
        gehe zum nächsten Abschnitt\;
    }{
        gehe zurück zum Beginn des aktuellen Abschnitts\;
    }
}
\caption{Wie man Algorithmen schreibt}
\end{algorithm}


%%%%%%%%%%%%%%%%%%%%%%%%%%%%%%%%%%%%%%%%%%%%%%%%%%%%%%
%%%%%%%%%%%%%%%%%%%%%%%%%%%%%%%%%%%%%%%%%%%%%%%%%%%%%%