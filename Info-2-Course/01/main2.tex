\documentclass[a4paper,12pt]{article}
\usepackage[margin=2.5cm]{geometry}
\setlength{\headheight}{1cm}
\usepackage{fancyhdr}
\usepackage[ngerman,german]{babel}
\usepackage[utf8]{inputenc}
\usepackage[ngerman,onelanguage]{algorithm2e}
\usepackage{amsmath}
\usepackage{amssymb}
\usepackage{amsthm}
\usepackage{mathtools}
\usepackage{enumerate}
\usepackage{hyperref}
\usepackage[german]{cleveref}


%%%%%%%%%%%%%%%%%%%%%%%%%%%%%%%%%%%%%%%%%%%%%%%%%%%%%%
%%%%%%%%%%%%%% EDIT THIS PART %%%%%%%%%%%%%%%%%%%%%%%%
%%%%%%%%%%%%%%%%%%%%%%%%%%%%%%%%%%%%%%%%%%%%%%%%%%%%%%
\newcommand{\Fach}{Algorithmen und Datenstrukturen}
\newcommand{\Name}{Ammar Saleh - 2329470} 
\newcommand{\name}{Humam Hamdan - 2427300}
\newcommand{\Gruppe}{4}
\newcommand{\Semester}{Sommersemester 2025}
\newcommand{\Uebungsblatt}{1} % UPDATE ME
%%%%%%%%%%%%%%%%%%%%%%%%%%%%%%%%%%%%%%%%%%%%%%%%%%%%%%
%%%%%%%%%%%%%%%%%%%%%%%%%%%%%%%%%%%%%%%%%%%%%%%%%%%%%%

%%%%%%%%%%%%%%%
%% Aufgaben-COMMAND
\newcommand{\Aufgabe}[1]{
  {
  \noindent\textsf{\textbf{Aufgabe #1}}
  \vspace*{0.2cm}
  }
}
%%%%%%%%%%%%%%
\hypersetup{
    pdftitle={\Fach{}: Uebungsblatt \Uebungsblatt{}},
    pdfauthor={Ammar Saleh und Humam Hamdan}
}


\title{Übungsblatt \Uebungsblatt}
\author{\Name}

\begin{document}
\thispagestyle{fancy}
\chead{\sf \large \Fach, \Semester}
\rhead{\sf \small \Name \\}
\lhead{\sf \small \name \\}
\vspace*{-1cm}
\begin{center}
\LARGE \sf \textbf{Übungsblatt \Uebungsblatt}
\end{center}

\Aufgabe{3}

\vspace{0.5cm}
\noindent
\textbf{Verfahren:}
\begin{itemize}
  \item \texttt{S.push(e)}:
  \begin{itemize}
    \item \texttt{enqueue(e)} in \(Q_2\)
    \item Übertrage alle Elemente aus \(Q_1\) nach \(Q_2\)
    \item Tausche \(Q_1 \leftrightarrow Q_2\)
  \end{itemize}
  \item \texttt{S.pop()}:
  \begin{itemize}
    \item \texttt{dequeue()} aus \(Q_1\)
  \end{itemize}
  \item \texttt{S.empty()}:
  \begin{itemize}
    \item Überprüfe, ob \(Q_1\) leer ist\\\\\\\\
  \end{itemize}
\end{itemize}

\subsubsection*{Beispiel:} 
\[
Q_1 = \emptyset, \quad Q_2 = \emptyset
\]

\subsubsection*{1. \texttt{S.push(l)}}

\[
\quad Q_1 = \emptyset,\; Q_2 = \fbox{l} \quad \rightarrow \quad Q_1 = \emptyset,\; Q_2 = \fbox{l} \quad \rightarrow \quad Q_1 = \fbox{l},\; Q_2 = \emptyset
\]

\subsubsection*{2. \texttt{S.push(a)}}

\[
Q_1 = \fbox{l},\; Q_2 = \fbox{a} \quad \rightarrow \quad Q_1 = \emptyset,\; Q_2 = \fbox{a} \rightarrow \fbox{l} \quad \rightarrow \quad Q_1 = \fbox{a} \rightarrow \fbox{l},\; Q_2 = \emptyset
\]

\subsubsection*{3. \texttt{S.pop()}}

\[
\text{Dequeue aus } Q_1 \rightarrow \fbox{a} \text{ wird entfernt}
\]
\[
Q_1 = \fbox{l}
\]

\subsubsection*{4. \texttt{S.empty()}}

\[
Q_1 = \fbox{l} \quad \rightarrow \quad \text{Stack ist \textbf{nicht leer}}
\]\\

So haben wir das Verhalten eines Stacks (LIFO) mithilfe von zwei Queues nachgebildet.

\end{document}
